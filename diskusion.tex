\section{Diskussion och slutsatser}
I detta kapitel knyter du ihop säcken.


\subsection{I resultatdiskusion / Diskusion av designprocessen}
Här diskuterar du dina analyserade resultat och utvärderar dem utifrån ditt syfte och dina frågeställningar. Här är det meningen att du skall reflektera över dina resultat, varför du fick just dessa resultat, hur de förhåller de sig till vad andra kommit fram till jämfört med din teoribakgrund.\index{teoribakgrund}\\ \\
Du som gör ett designarbete diskuterar här din egen designprocess (eftersom den är en viktig del av ”resultatet” i ett examensarbete) och hur slutprodukten blev i förhållande till vad du tänkt dig enligt syfte och frågeställningar.\\ \\
Ett bra tips för att få en bra struktur på detta avsnitt och som samtidigt hjälper dig att hålla en röd tråd i rapporen är att du inleder med att repetera ditt syfte och sedan utgår från dina frågeställningar som rubriker (det kan du även göra i föregående kapitel om du vill). Då underlättar du för både dig själv och läsaren – det hjälper dig att se till att du verkligen besvarar dina frågeställningar och uppfyller syftet med examensarbetet och det blir lätt för läsaren att följa hur du gjort detta.

\subsection{Metoddiskussion}
Du inleder med en Metoddiskussion under egen rubrik. Här diskuterar du dina metodval och genomförandet i stort och de styrkor/svagheter som du upplever finns i detta avseende. Diskutera hur väl du tycker att du lyckades åstadkomma det du avsåg utifrån ditt syfte och dina frågeställningar genom dina metodval och tillvägagångssätt. Vad fungerade bra/mindre bra? Vad kunde du ha gjort bättre? Hur väl har du uppfyllt kraven på validitet och reliabilitet?

\subsection{Slutsatser och rekommendationer}
Under rubriken Slutsatser summerar du kortfattat de viktigaste huvudpunkterna.\\ \\
Avslutningsvis bör du ge förslag och rekommendationer till hur arbetet skulle kunna fortsättas/utvecklas ytterligare.