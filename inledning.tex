\section{Inleding}
Inledningen ska sätta in läsaren i det sammanhang \index{sammanhang} arbetet har utförts och även beskriva att examensarbetet har genomförts som en del i din utbildning. Inledningen får inte bli för lång. Tänk på att det är den inledande texten som ska fånga din läsare. Alltför detaljerad information ska inte placeras här.

\subsection{Bakgrund och problembeskrivning}
Smalna av din beskrivning så att du i slutet av bakgrundsbeskrivningen\index{bakgrundsbeskrivningen} är framme vid den problembeskrivning\index{problembeskrivning} som ligger till grund för syftet och de frågeställningar som du valt att arbeta vidare med och som du beskriver mer exakt i nästa avsnitt, dvs. tänk trattformat när du skriver detta avsnitt och landa i problembeskrivningen. Läsaren ska utifrån bakgrundsbeskrivningen få grepp om varför det är relevant och intressant att göra ett examensarbete utifrån den problembeskrivning du valt. \\ \\
För dig som arbetar med design av produkter/system ska du i detta avsnitt beskriva hur produkten/systemet används och hanteras idag och vilka brister/fel som produkten/systemet är belastat med idag. (Exempelvis: Problemet med dagens lösning är att handtaget på maskinen är utformat för stort och därmed inte kan användas av personer med mindre handstorlek. Ett annat problem är placeringen av...).

\subsection{Syfte och frågeställningar}
Ange klart och koncist syfte och frågeställningar\index{frågeställningar} . Syftet definierar vad det är som ska utföras, utredas eller sammanställas samt vad som är nyttan med det. Syftet bryts ner i ett antal frågeställningar (max 3, i undantagsfall 4) som du kommer att besvara i rapporten för att kunna uppfylla syftet. Det är ditt syfte och dina frågeställningar som vägleder dig genom hela arbetet. Det är därför viktigt att du ägnar mycket tid till att tänka igenom och klargöra för dig själv tillsammans med din handledare och en eventuell uppdragsgivare vad det är du ska uppnå. För dig som arbetar med design av produkter/system går det vanligtvis också att formulera i syfte och frågeställningar. Tänk på att syftet som anges här kan skilja sig från målet med ditt uppdrag, speciellt när det gäller examensarbeten inom designområdet. Målet kan vara en produkt medan syftet är att redogöra för hur du gått tillväga för att designa/utforma produkten enligt vissa kriterier, dvs. processen att ta fram en produkt är då det viktigaste att beskriva i rapporten.

\subsection{Avgränsningar}
Här ska du komplettera syfte och frågeställningar med att beskriva vad som inbegrips i studien men framförallt vad som inte omfattas. Observera att du inte ska upprepa syftet här utan komplettera med information som förtydligar hur du avgränsar ditt arbete. Ange vilka avgränsningar \index{avgränsningar}  du gör för att arbetet inte ska bli för stort och rapporten inte ska bli för omfattande.

\subsection{Disposition}
Beskriv hur resten av rapporten är disponerad, alltså kort hur rapporten är uppbyggd för att hjälpa läsaren att få en bild av rapportens upplägg.